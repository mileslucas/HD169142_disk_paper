% Define document class
\documentclass[twocolumn]{aastex631}
\usepackage{showyourwork}
\usepackage{multirow}
\usepackage{CJK}
\usepackage{datetime}
\yyyymmdddate
\usepackage{mathtools}
\usepackage{savesym}
\let\tablenum\relax
\usepackage{siunitx}[=v2]
% redefine deluxetable for compatibility with the array package.
\let\oldenddeluxetable\endxdeluxetable
\let\olddeluxetable\xdeluxetable
\makeatletter
\renewenvironment{xdeluxetable}[1]{
\olddeluxetable{[#1]}
\def\pt@format{\string#1}%
}{\oldenddeluxetable}
\makeatother

\sisetup{separate-uncertainty=true}

\DeclareSIUnit{\mag}{mag}
\DeclareSIUnit{\mas}{mas}
\DeclareSIUnit{\electron}{e^-}
\DeclareSIUnit{\pixel}{px}
\DeclareSIUnit{\adu}{adu}
\DeclareSIUnit{\year}{yr}
\DeclareSIUnit{\parsec}{pc}
\DeclareSIUnit{\au}{au}
% Begin!
\begin{document}

% Title
\title{A protoplanet in its early stages of formation: a decadal analysis of the HD 169142 transition disk}
\shorttitle{A Decadal Analysis of HD 169142}

% Author list
% Author list
\author[0000-0001-6341-310X]{Miles Lucas}
\affiliation{Institute for Astronomy, University of Hawai'i, 640 N. Aohoku Pl., Hilo, HI 96720, USA}
\affiliation{Subaru Telescope, National Astronomical Observatory of Japan, 650 N. Aohoku Pl., Hilo, HI 96720, USA}

\author[0000-0003-1341-5531]{Michael Bottom}
\affiliation{Institute for Astronomy, University of Hawai'i, 640 N. Aohoku Pl., Hilo, HI 96720, USA}

\author{Ruobing Dong}
\affiliation{china}
\affiliation{University of Victoria}

\author{Myriam Benisty}
\affiliation{Max Planck Institute for Astronomy}
\author{Mario Flock}
\affiliation{Max Planck Institute for Astronomy}

\author{Jonathan Williams}
\affiliation{Institute for Astronomy, University of Hawai'i, 2680 Woodlawn Dr., Honolulu, HI 96826, USA}


% alphabetical by last name after this
\author[0000-0002-1094-852X]{Kyohoon Ahn}
\affiliation{Subaru Telescope, National Astronomical Observatory of Japan, 650 N. Aohoku Pl., Hilo, HI 96720, USA}
\affiliation{Korea Astronomy and Space Science Institute, 776 Daedeok-daero, Yuseong-gu, Daejeon 34055, Republic of Korea}




\author[0000-0003-4514-7906]{Vincent Deo}
\affiliation{Subaru Telescope, National Astronomical Observatory of Japan, 650 N. Aohoku Pl., Hilo, HI 96720, USA}

\author[0000-0002-7405-3119]{Thayne Currie}
\affiliation{Subaru Telescope, National Astronomical Observatory of Japan, 650 N. Aohoku Pl., Hilo, HI 96720, USA}
\affiliation{Department of Physics and Astronomy, University of Texas at San Antonio, One UTSA Circle, San Antonio, TX 78249, USA}



\author[0000-0002-1097-9908]{Olivier Guyon}
\affiliation{Subaru Telescope, National Astronomical Observatory of Japan, 650 N. Aohoku Pl., Hilo, HI 96720, USA}
\affiliation{College of Optical Sciences, University of Arizona, Tucson, AZ 87521, USA}
\affiliation{Steward Observatory, University of Arizona, Tucson, AZ 87521, USA}
\affiliation{Astrobiology Center, 2 Chome-21-1, Osawa, Mitaka, Tokyo, 181-8588, Japan}



\author[0000-0002-9294-1793]{Tomoyuki Kudo}
\affiliation{Subaru Telescope, National Astronomical Observatory of Japan, 650 N. Aohoku Pl., Hilo, HI 96720, USA}

\author[0000-0002-3047-1845]{Julien Lozi}
\affiliation{Subaru Telescope, National Astronomical Observatory of Japan, 650 N. Aohoku Pl., Hilo, HI 96720, USA}

\author[0000-0001-6205-9233]{Maxwell Millar-Blanchaer}
\affiliation{Department of Physics, University of California, Santa Barbara, CA, 93106, USA}

\author[0000-0003-1713-3208]{Boris Safonov}
\affiliation{Sternberg Astronomical Institute, Lomonosov Moscow State Univeristy, 119992 Universitetskii prospekt 13, Moscow, Russia}

\author[0000-0002-6879-3030]{Taichi Uyama}
\affiliation{Department of Physics and Astronomy, California State University, Northridge, Northridge, CA 91330 USA}

\author[0000-0003-4018-2569]{S\'ebastien Vievard}
\affiliation{Subaru Telescope, National Astronomical Observatory of Japan, 650 N. Aohoku Pl., Hilo, HI 96720, USA}
\affiliation{Astrobiology Center, 2 Chome-21-1, Osawa, Mitaka, Tokyo, 181-8588, Japan}


\author[0000-0003-3567-6839]{Manxuan Zhang}
\affiliation{Department of Physics, University of California, Santa Barbara, CA, 93106, USA}



% Abstract with filler text
\begin{abstract}
\end{abstract}

\section{Introduction\label{sec:introduction}}


Exoplanet detection and characterization is one of the frontiers in modern observational astronomy, driven by cutting-edge instrumentation. In recent decades, thousands of exoplanets have been discovered, revealing a great diversity of planetary systems and architectures. However, the understanding of how these systems formed and evolved remains limited, with discrepancies between the observed planets and theories of planet formation. 

Direct observations of protoplanetary disks offer the best way to study planet formation by resolving features like spirals, ring gaps, and clumps \citep{andrews_observations_2020,benisty_optical_2023}. These features indicate over- and under-densities of disk material, which can be signatures of planet formation.

Major surveys have already detected most visibly bright disks ($m_{\text R}<$12) in nearby star-forming regions ($d <$ \SIrange{150}{400}{\parsec}; e.g., \citealt{avenhaus_disks_2018,garufi_disks_2020,rich_gemini-lights_2022,ren_protoplanetary_2023,valegard_sphere_2024,ginski_sphere_2024,garufi_sphere_2024}). Further advancements in disk science require shifting from detection to in-depth characterization.

High-contrast polarimetric imaging is the most powerful tool for studying planet-forming disks at optical to near-infrared wavelengths \citep{benisty_optical_2023}. This technique uses adaptive optics correction, large-diameter telescopes, coronagraphs for starlight suppression, polarization optics, and low-noise detectors to achieve the angular resolution and sensitivity required for detecting faint disk structures close to the intense diffracted light of the star.

The key advantage of polarimetry is that starlight reflected off disk material is partially polarized (20\% - 70\% at peak), whereas the star is unpolarized. Using a polarimeter to split and then subtract orthogonal polarization states cancels out the unpolarized stellar light to reveal the disk, known as polarimetric differential imaging (PDI; \citealt{kuhn_imaging_2001}). This cancellation attenuates the stellar signal without biases that alter the morphology and photometry of the disk, unlike typical point-spread function (PSF) subtraction techniques \citep{soummer_detection_2012,benisty_optical_2023}.

In this paper, we present dynamical analysis of polarimetric images of the planet-forming disk HD169142. We will present a combination of archival PDI images and new observations with scexao (\autoref{sec:observations}). We present the observations 
\section{Observations\label{sec:observations}}

In this work, we present a combination of new observations of HD 169142 using SCExAO/VAMPIRES alongside previously published, archival datasets from various high-contrast instruments. All observations are presented in the observing log (\autoref{tbl:obslog}).


\begin{deluxetable*}{llllccccccl}
\tabletypesize{\footnotesize}
\tablehead{
    \colhead{Date} &
    \multirow{2}{*}{Telescope} &
    \multirow{2}{*}{Instrument} &
    % \multirow{2}{*}{Object} &
    \multirow{2}{*}{Filter} &
    \colhead{IWA} &
    \colhead{pix. scale} &
    \colhead{DIT} &
    \colhead{$T_\mathrm{exp}$} &
    \colhead{Seeing} &
    \multirow{2}{*}{Ref.} \vspace{-0.75em} \\
    \colhead{(UTC)} & % date
    & % inst
    & % object
    & % filter
    \colhead{(mas)} & % iwa
    \colhead{(mas/pix)} & % pxscale
    \colhead{(\si{\second})} & % dit
    \colhead{(\si{min})} & %texp
    \colhead{('')} & % seeing
     & % reference
}
\tablecaption{Chronologically-ordered observing log.\label{tbl:obslog}}
\startdata
\formatdate{26}{07}{2012} & VLT & NACO &  H & - & 27 & & & & [1] \\
\formatdate{25}{04}{2014} & Gemini-S & GPI &  J & 184 & 14.14 & 29.1 & 62 & \num{0.85\pm0.27} & [2,3] \\
\formatdate{03}{05}{2015} & VLT & IRDIS &  J & 80 & 12.25 & & & 0.9 & [4] \\ 
\formatdate{10}{07}{2015} & VLT & ZIMPOL &  VBB & - & 3.6 & & & & [5] \\
\formatdate{18}{09}{2017} & ALMA & - &  \SI{1.3}{\milli\meter} & - & 5 & - & - & - & [6] \\
\formatdate{15}{07}{2018} & VLT & ZIMPOL &  VBB & - & 3.6 & & & & [7]  \\
\formatdate{06}{09}{2021} & VLT & IRDIS &  Ks & 200 & 12.25 & 16 & 34 & & [8] \\
\formatdate{1}{6}{2023} & Subaru & CHARIS &  JHK & 113 & 15.16 &  &  &  & \\ 
\formatdate{2}{6}{2023} & Subaru & CHARIS &  JHK & 113 & 15.16 &  &  &  & \\ 
\formatdate{3}{6}{2023} & Subaru & CHARIS &  JHK & 113 & 15.16 &  &  &  & \\ 
\formatdate{4}{6}{2023} & Subaru & CHARIS &  JHK & 113 & 15.16 &  &  &  & \\ 
% \formatdate{7}{7}{2023} & Subaru & VAMPIRES &  MBI & 109 & 5.9 & 1 & 50 & \num{0.83\pm0.21} & [9] \\
\formatdate{29}{7}{2024} & Subaru & VAMPIRES & MBI & 54 & 5.9 & 2 & 108 & \num{0.6\pm0.14} & \\
\enddata
\tablecomments{IWA: inner working angle of coronagraph, if present. DIT: detector integration time for each frame. $T_\text{exp}$ total exposure time. Seeing estimates, where present, were retrieved from the reference publication for archival data and the mean and standard deviation DIMM from the Maunakea Weather Center archive for the VAMPIRES observations.}
\tablerefs{[1]: \citealt{quanz_gaps_2013}, [2]: \citealt{monnier_polarized_2017}, [3]: \citealt{rich_gemini-lights_2022}, [4]: \citealt{pohl_circumstellar_2017}, [5]: \citealt{bertrang_hd_2018}, [6]: \citealt{perez_dust_2019}, [7]: \citealt{bertrang_moving_2020}, [8]: \citealt{ren_protoplanetary_2023}}
\end{deluxetable*}

\subsection{SCExAO/CHARIS Observations\label{sec:obs_charis}}

We observed HD 169142 over four nights in 2023 using the CHARIS integral-field spectrograph \citep{groff_charis_2015,groff_first_2017} in broadband PDI mode. In this mode, the combination of a dispersing prism and microlens array creates spectra across the field of view from J to K band. With the Wollaston prism both the ordinary and extraordinary beams are imaged and a rectangular \ang{;;2}x\ang{;;1} field stop prevents overlap of the beams. An upstream half-wave plate (HWP) is modulated to measure the linear Stokes parameters and correct for instrumental polarization. We used the \SI{113}{mas} IWA Lyot coronagraph with calibration ``satellite spot'' speckles for alignment and photometric calibration \citep{sahoo_precision_2020}.

Spectral cubes were extracted from each night's data using the \texttt{CHARIS-DRP}. Then, the spectral cubes were post-processed to do sky subtraction, image registration, spectrophotometric calibration, and PDI using the \texttt{CHARIS-DPP}. The data was derotated and a Mueller-matrix correction was applied to the resulting Stokes Q and U images for each wavelength slice \citep{joost_t_hart_full_2021}. The individual Stokes Q and U images from all four nights were combined, assuming there will be negligible disk motion over four days. Finally, for the Q and U images all wavelength slices, excluding the telluric absorption bands, were mean-combined. We calculated the azimuthal Stokes $Q_\phi$ and $U_\phi$ \citep{schmid_limb_2006,monnier_multiple_2019},
\begin{align}
\begin{split}
    \label{eqn:az_stokes}
    Q_\phi &= -Q\cos{\left(2\phi\right)} - U\sin{\left(2\phi\right)} \\
    U_\phi &= Q\sin{\left(2\phi\right)} - U\cos{\left(2\phi\right)},
\end{split}
\end{align}
where
\begin{equation}
    \phi = \arctan{\left( \frac{x_\star - x}{y - y_\star} \right)} + \phi_0.
\end{equation}

\subsection{SCExAO/VAMPIRES Observations\label{sec:obs_vampires}}

We observed HD 169142 in 2024 using VAMPIRES \citep{norris_vampires_2015,lucas_visible-light_2024-1} on the Subaru Coronagraphic Extreme Adaptive Optics instrument \citep{jovanovic_subaru_2015}. Our observations utilized the new multiband imaging mode (MBI), which produces four broadband images simultaneously. We used the slow detector readout mode for low read noise and the ``slow'' polarimetry mode, which only uses the upstream HWP for modulation to correct for instrumental polarization. We used the \SI{54}{mas} IWA Lyot coronagraph with calibration speckles for alignment and photometric calibration.

We processed the data using the VAMPIRES data processing pipeline\footnote{\url{https://github.com/scexao-org/vampires_dpp}}. The processing steps included dark subtraction, sub-cropping the four MBI fields, frame registration, frame selection, coadding, photometric calibration, and polarimetric differential imaging (PDI). We registered the frames using sub-pixel phase cross-correlation \citep{guizar-sicairos_efficient_2008} with the mean frame of each data cube. We discarded the lowest 25\% of frames from each cube according to the estimated Strehl ratio before median-collapsing the cube into a single frame. The coadded frames are grouped according to their HWP angle, with incomplete groupings discarded. We performed double-difference imaging with a Mueller matrix correction of each group to produce calibrated Stokes I, Q, and U frames for each wavelength. The four wavelengths were mean-combined before calculating the $Q_\phi$ and $U_\phi$ images.

\subsection{Archival Data\label{sec:obs_archive}}

HD 169142 was observed by VLT/NACO in 2012 using its H-band polarimetric mode \citep{quanz_very_2011}. The original data is described in detail in \citet{quanz_gaps_2013}; however, we have used the newly-processed data products from the \texttt{PIPPIN} pipeline \citep{regt_polarimetric_2024}, which uses modern polarimetric data processing techniques in a consistent pipeline purpose-built for NACO. 

The Gemini Planet Imager (GPI; \citealt{macintosh_first_2014}) observed HD 169142 in 2014 using its J-band non-dispersed polarimetric mode. A 999 mas IWA apodized Lyot coronagraph was used for diffraction control. The data is presented in \citet{monnier_polarized_2017} and \citet{rich_gemini-lights_2022}. 

VLT/SPHERE \citep{beuzit_sphere_2019} has observed HD 169142 on numerous occasions with both IRDIS \citep{boer_polarimetric_2020,holstein_polarimetric_2020} and ZIMPOL \citep{schmid_spherezimpol_2018}. In 2015 it was observed with IRDIS at J-band \citep{pohl_circumstellar_2017} and ZIMPOL in their VBB filter, which roughly spans R- to I-band \citep{bertrang_hd_2018,tschudi_quantitative_2021}. In 2018, it was observed by ZIMPOL, again, in the VBB filter \citep{bertrang_moving_2020}. In 2021, HD 169142 was observed with IRDIS at K-band using star-hopping \citep{ren_protoplanetary_2023}.

Lastly, we obtained the high-resolution ALMA \SI{1.3}{\milli\meter} continuum images presented in \citet{perez_dust_2019}.


\input{sections/03_polarimetric_images}
\input{sections/04_keplerian_analysis}
\input{sections/05_section5}

\section*{Acknowledgements}
First and foremost, we wish to recognize and acknowledge the significant cultural role and reverence that the summit of Maunakea has always had within the Indigenous Hawaiian community. We are grateful and thank the community for the privilege of conducting observations from this sacred mauna.

We thank telescope operators TODO for their support during the observations used in this work. We thank Gesa Bertrang and Mario Flock for sharing the ZIMPOL data used in this work. We thank Myriam Benisty for sharing the IRDIS data used in this work.

This research was funded by the Heising-Simons Foundation through grant \#2020-1823. Based on data collected at Subaru Telescope, which is operated by the National Astronomical Observatory of Japan. The development of SCExAO is supported by the Japan Society for the Promotion of Science (Grant-in-Aid for Research \#23340051, \#26220704, \#23103002, \#19H00703, \#19H00695, and \#21H04998), the Subaru Telescope, the National Astronomical Observatory of Japan, the Astrobiology Center of the National Institutes of Natural Sciences, Japan, the Mt Cuba Foundation and the Heising-Simons Foundation.

\bibliography{references}

\end{document}
