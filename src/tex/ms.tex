\documentclass[twocolumn]{aastex631}
\usepackage{showyourwork}
\usepackage{multirow}
\usepackage{datetime}
\yyyymmdddate
\usepackage{mathtools}
\usepackage{savesym}
\let\tablenum\relax
\usepackage{siunitx}[=v2]
% redefine deluxetable for compatibility with the array package.
\let\oldenddeluxetable\endxdeluxetable
\let\olddeluxetable\xdeluxetable
\makeatletter
\renewenvironment{xdeluxetable}[1]{
\olddeluxetable{[#1]}
\def\pt@format{\string#1}%
}{\oldenddeluxetable}
\makeatother

\sisetup{separate-uncertainty=true}

\DeclareSIUnit{\mag}{mag}
\DeclareSIUnit{\mas}{mas}
\DeclareSIUnit{\electron}{e^-}
\DeclareSIUnit{\pixel}{px}
\DeclareSIUnit{\adu}{adu}
\DeclareSIUnit{\year}{yr}
\DeclareSIUnit{\parsec}{pc}
\DeclareSIUnit{\au}{au}
% Begin!
\begin{document}

% Title
\title{HD 169142}

% Author list
% Author list
\author[0000-0001-6341-310X]{Miles Lucas}
\affiliation{Institute for Astronomy, University of Hawai'i, 640 N. Aohoku Pl., Hilo, HI 96720, USA}
\affiliation{Subaru Telescope, National Astronomical Observatory of Japan, 650 N. Aohoku Pl., Hilo, HI 96720, USA}

\author[0000-0003-1341-5531]{Michael Bottom}
\affiliation{Institute for Astronomy, University of Hawai'i, 640 N. Aohoku Pl., Hilo, HI 96720, USA}

\author[0000-0001-9290-7846]{Ruobing Dong (董若冰)}
\affiliation{Kavli Institute for Astronomy and Astrophysics, Peking University, Beijing 100871, China}
\affiliation{Department of Physics and Astronomy, University of Victoria, Victoria, BC, V8P 5C2, Canada}

\author[0000-0002-7695-7605]{Myriam Benisty}
\affiliation{Max-Planck-Institut f\"ur Astronomie, K\"onigstuhl 17, D-67117 Heidelberg, Germany}


\author[0000-0002-9298-3029]{Mario Flock}
\affiliation{Max-Planck-Institut f\"ur Astronomie, K\"onigstuhl 17, D-67117 Heidelberg, Germany}

\author[0000-0001-5058-695X]{Jonathan Williams}
\affiliation{Institute for Astronomy, University of Hawai'i, 2680 Woodlawn Dr., Honolulu, HI 96826, USA}

\author{Maria Vincent}
\affiliation{Institute for Astronomy, University of Hawai'i, 2680 Woodlawn Dr., Honolulu, HI 96826, USA}

% alphabetical by last name after this
\author[0000-0002-1094-852X]{Kyohoon Ahn}
\affiliation{Subaru Telescope, National Astronomical Observatory of Japan, 650 N. Aohoku Pl., Hilo, HI 96720, USA}
\affiliation{Korea Astronomy and Space Science Institute, 776 Daedeok-daero, Yuseong-gu, Daejeon 34055, Republic of Korea}




\author[0000-0003-4514-7906]{Vincent Deo}
\affiliation{Subaru Telescope, National Astronomical Observatory of Japan, 650 N. Aohoku Pl., Hilo, HI 96720, USA}

\author[0000-0002-7405-3119]{Thayne Currie}
\affiliation{Subaru Telescope, National Astronomical Observatory of Japan, 650 N. Aohoku Pl., Hilo, HI 96720, USA}
\affiliation{Department of Physics and Astronomy, University of Texas at San Antonio, One UTSA Circle, San Antonio, TX 78249, USA}



\author[0000-0002-1097-9908]{Olivier Guyon}
\affiliation{Subaru Telescope, National Astronomical Observatory of Japan, 650 N. Aohoku Pl., Hilo, HI 96720, USA}
\affiliation{College of Optical Sciences, University of Arizona, Tucson, AZ 87521, USA}
\affiliation{Steward Observatory, University of Arizona, Tucson, AZ 87521, USA}
\affiliation{Astrobiology Center, 2 Chome-21-1, Osawa, Mitaka, Tokyo, 181-8588, Japan}



\author[0000-0002-9294-1793]{Tomoyuki Kudo}
\affiliation{Subaru Telescope, National Astronomical Observatory of Japan, 650 N. Aohoku Pl., Hilo, HI 96720, USA}

\author[0000-0002-3047-1845]{Julien Lozi}
\affiliation{Subaru Telescope, National Astronomical Observatory of Japan, 650 N. Aohoku Pl., Hilo, HI 96720, USA}

\author[0000-0001-6205-9233]{Maxwell Millar-Blanchaer}
\affiliation{Department of Physics, University of California, Santa Barbara, CA, 93106, USA}

\author[0000-0003-1713-3208]{Boris Safonov}
\affiliation{Sternberg Astronomical Institute, Lomonosov Moscow State Univeristy, 119992 Universitetskii prospekt 13, Moscow, Russia}

\author[0000-0002-6879-3030]{Taichi Uyama}
\affiliation{Department of Physics and Astronomy, California State University, Northridge, Northridge, CA 91330 USA}

\author[0000-0003-4018-2569]{S\'ebastien Vievard}
\affiliation{Subaru Telescope, National Astronomical Observatory of Japan, 650 N. Aohoku Pl., Hilo, HI 96720, USA}
\affiliation{Astrobiology Center, 2 Chome-21-1, Osawa, Mitaka, Tokyo, 181-8588, Japan}


\author[0000-0003-3567-6839]{Manxuan Zhang}
\affiliation{Department of Physics, University of California, Santa Barbara, CA, 93106, USA}



% Abstract with filler text
\begin{abstract}
\end{abstract}

\section{Introduction\label{sec:introduction}}

\begin{figure}
    \centering
    \script{plot_disk_schematic.py}
    \includegraphics[width=\columnwidth]{figures/HD169142_schematic.pdf}
    \caption{Schematic diagram of the HD169142 transitional disk. The black contours are based on the ALMA 1.3mm continuum data which trace out an inner disk (B0) a close inner ring (B1) and three outer rings (B2-B4). The red. contours are based on visible to near-IR scattered-light images, separated into an inner and outer ring.\label{fig:disk_schematic}}
\end{figure}

\begin{deluxetable}{lcll}
    \centering
    \tablecaption{HD 169142 adopted parameters.\label{tbl:system}}
    \tablehead{
        \colhead{Parameter} &
        \colhead{Value} & 
        \colhead{Unit} & 
        \colhead{Ref.}
    }
    \startdata
    \cutinhead{System parameters}
    Distance & \num{114.87\pm0.35} & \si{pc} & [1] \\
    $M_\star$ & 1.65 & $M_\odot$ & \\
    \cutinhead{Disk parameters}
    Inclination & \num{12.5} & deg &  \\
    Position angle & \num{5} & deg &  \\
    \enddata
    \tablerefs{[1]: \citealt{gaia_collaboration_gaia_2021}}
\end{deluxetable}
\section{Observations\label{sec:observations}}

In this work, we present a combination of new observations of HD 169142 using SCExAO alongside previously published, archival datasets from various high-contrast instruments. All observations are presented in the observing log (\autoref{tbl:obslog}).


\begin{deluxetable*}{llllccccccl}
\tabletypesize{\footnotesize}
\tablehead{
    \colhead{Date} &
    \multirow{2}{*}{Telescope} &
    \multirow{2}{*}{Instrument} &
    % \multirow{2}{*}{Object} &
    \multirow{2}{*}{Filter} &
    \colhead{IWA} &
    \colhead{pix. scale} &
    \colhead{DIT} &
    \colhead{$T_\mathrm{exp}$} &
    \colhead{Seeing} &
    \multirow{2}{*}{Ref.} \vspace{-0.75em} \\
    \colhead{(UTC)} & % date
    & % inst
    & % object
    & % filter
    \colhead{(mas)} & % iwa
    \colhead{(mas/pix)} & % pxscale
    \colhead{(\si{\second})} & % dit
    \colhead{(\si{min})} & %texp
    \colhead{('')} & % seeing
     & % reference
}
\tablecaption{Chronologically-ordered observing log.\label{tbl:obslog}}
\startdata
\formatdate{26}{07}{2012} & VLT & NACO & H & sat. & 27 & 45 & 69 & $\sim$1.04 & [1] \\
\formatdate{25}{04}{2014} & Gemini-S & GPI & J & 184 & 14.14 & 29.1 & 62 & \num{0.85\pm0.27} & [2] \\
\formatdate{03}{05}{2015} & VLT & IRDIS & J & 80 & 12.263 & 16 & 53 & $\sim$0.9 & [3] \\ 
\formatdate{10}{07}{2015} & VLT & ZIMPOL & VBB & sat. & 3.6 & 10 & 56 & 0.75 & [4] \\
% \formatdate{18}{09}{2017} & ALMA & - &  \SI{1.3}{\milli\meter} & - & 5 & - & - & - & [6] \\
\formatdate{15}{07}{2018} & VLT & ZIMPOL & VBB & sat. & 3.6 & 4.4 & 53 & 0.51 & [5]  \\
\formatdate{06}{09}{2021} & VLT & IRDIS & Ks & 200 & 12.265 & 16 & 34 & $\sim$0.5 & [6] \\
\formatdate{3}{6}{2023} & Subaru & CHARIS & JHK & 113 & 15.16 & 60 & 52 & \num{0.82\pm0.15}$^a$ & \\ 
\formatdate{4}{6}{2023} & Subaru & CHARIS & JHK & 113 & 15.16 & 60 & 40 &  & \\ 
% \formatdate{7}{7}{2023} & Subaru & VAMPIRES &  MBI & 109 & 5.9 & 1 & 50 & \num{0.83\pm0.21} & [9] \\
\formatdate{29}{7}{2024} & Subaru & VAMPIRES & MBI & 54 & 5.9 & 2 & 108 & \num{0.6\pm0.14}$^a$ & \\
\enddata
\tablecomments{IWA: inner working angle of coronagraph, if present, or ``sat.'' for saturated non-coronagraphic observations. DIT: detector integration time for each frame. $T_\text{exp}$ total exposure time. (a): Seeing estimates, where present, were retrieved from the mean and standard deviation DIMM from the Maunakea Weather Center archive (\url{http://mkwc.ifa.hawaii.edu/current/seeing/index.cgi}).}
\tablerefs{[1]: \citealt{quanz_gaps_2013}, [2]: \citealt{monnier_polarized_2017}, [3]: \citealt{pohl_circumstellar_2017}, [4]: \citealt{bertrang_hd_2018}, [5]: \citealt{bertrang_moving_2020}, [8]: \citealt{ren_protoplanetary_2023}}


\end{deluxetable*}
\subsection{SCExAO/CHARIS Observations\label{sec:obs_charis}}

We observed HD 169142 over two nights in 2023 using the CHARIS integral-field spectrograph in broadband PDI mode \citep{groff_charis_2015,groff_first_2017}. The combination of a dispersing prism and microlens array creates spectra across the field of view from J to K band (\SIrange{1.12}{2.4}{\micro\meter}) and the Wollaston prism splits the ordinary and extraordinary beams which are simultaneously imaged on the detector. A rectangular \ang{;;2}x\ang{;;1} field stop is used to prevent the two orthogonal beams from overlapping creating a distinct wedge in derotated and collapsed data. An upstream half-wave plate (HWP) is modulated to measure the linear Stokes parameters ($Q$ and $U$) and correct for instrumental polarization. We used the \SI{113}{mas} IWA Lyot coronagraph with calibration ``satellite spot'' speckles for alignment and photometric calibration \citep{sahoo_precision_2020}.

Spectral cubes were extracted from each night's data using \texttt{CHARIS-DEP} \citep{brandt_data_2017}. Then, the spectral cubes were post-processed with sky subtraction, image registration, spectrophotometric calibration, and PDI using the \texttt{CHARIS-DPP} \citep{currie_subaruscexao_2017,lawson_high-contrast_2021}. The data was derotated and a Mueller-matrix correction was applied to the resulting Stokes Q and U images for each wavelength slice \citep{joost_t_hart_full_2021}. The individual Stokes Q and U images from both nights were combined, assuming negligible disk motion. Finally, for the Q and U images all wavelength slices, excluding the telluric absorption bands, were combined with an outlier-resilient mean. Some of the individual Q and U frames show that the coronagraphic mask was misaligned and partially obscuring the inner ring to the east. This only occurs in a few frames but does appear as a slight dip in intensity in the inner ring profiles. We calculated the azimuthal Stokes $Q_\phi$ and $U_\phi$ \citep{schmid_limb_2006,monnier_multiple_2019},
\begin{align}
\begin{split}
    \label{eqn:az_stokes}
    Q_\phi &= -Q\cos{\left(2\phi\right)} - U\sin{\left(2\phi\right)} \\
    U_\phi &= -Q\sin{\left(2\phi\right)} + U\cos{\left(2\phi\right)},
\end{split}
\end{align}
where
\begin{equation}
    \phi = \arctan{\left( \frac{x_\star - x}{y - y_\star} \right)} + \phi_0,
\end{equation}
the angle East of North plus any calibration offsets.

\subsection{SCExAO/VAMPIRES Observations\label{sec:obs_vampires}}

We observed HD 169142 in 2024 using SCExAO/VAMPIRES \citep{norris_vampires_2015,lucas_visible-light_2024-1}. Our observations utilized the new multiband imaging mode (MBI), which produces four broadband images simultaneously. We used the slow detector readout mode for low read noise and the ``slow'' polarimetry mode, which only uses the upstream HWP for modulation to correct for instrumental polarization. We used the \SI{54}{mas} IWA Lyot coronagraph with calibration speckles for alignment and photometric calibration.

We processed the data using the VAMPIRES data processing pipeline\footnote{\url{https://github.com/scexao-org/vampires_dpp}}. The processing steps included dark subtraction, sub-cropping the four MBI fields, frame registration, frame selection, coadding, photometric calibration, and polarimetric differential imaging (PDI). We registered the frames using sub-pixel phase cross-correlation \citep{guizar-sicairos_efficient_2008} with the mean frame of each data cube. We discarded the lowest 25\% of frames from each cube according to the estimated Strehl ratio before median-collapsing the cube into a single frame. The coadded frames were grouped according to their HWP angle, with incomplete groupings discarded. We performed double-difference imaging with a Mueller-matrix correction of each group to produce calibrated Stokes I, Q, and U frames for each wavelength. The four wavelengths were mean-combined before calculating the $Q_\phi$ and $U_\phi$ images.

\subsection{Archival Data\label{sec:obs_archive}}

The oldest dataset is the H-band polarimetric observation on the Very Large Telescope (VLT) Nasmyth Adaptive Optics System Near-Infrared Imager and Spectrograph (NACO; \citealp{quanz_very_2011}) presented in \citet{quanz_gaps_2013}. The data is non-coronagraphic, however, the central star is saturated and masked out in the images. We used the data presented in \citet{regt_polarimetric_2024} from the \texttt{PIPPIN} pipeline, which uses modern polarimetric data processing techniques in a consistent, purpose-built package for NACO.

The Gemini Planet Imager (GPI; \citealt{macintosh_first_2014}) observed HD 169142 in 2014 in J-band non-dispersed polarimetric mode \citep{perrin_polarimetry_2015}. A 184 mas IWA apodized Lyot coronagraph (APLC) was used for diffraction control. The data was originally presented in \citet{monnier_polarized_2017} and we use the data products from \citet{rich_gemini-lights_2022}.

The Spectro-Polarimetic High contrast imager for Exoplanets REsearch (SPHERE; \citealp{beuzit_sphere_2019}) has observed HD 169142 on numerous occasions with the InfraRed Dual Imager and Spectrograph (IRDIS; \citealp{dohlen_infra-red_2008}) and the Zurich IMaging POLarimeter (ZIMPOL; \citealp{schmid_spherezimpol_2018}). In 2015 it was observed with IRDIS in J-band polarimetric mode \citep{boer_polarimetric_2020} with an \SI{80}{\mas} IWA APLC, first presented in \citet{pohl_circumstellar_2017}. The data was processed with \texttt{IRDAP} \citep{holstein_polarimetric_2020} using the calibrated Mueller-matrix correction. HD 169142 was also observed in 2021 in Ks-band polarimetric mode with a \SI{200}{\mas} IWA APLC, first presented in \citet{ren_protoplanetary_2023}. The first ZIMPOL observation was in 2015 and 2018 using the very broadband (VBB) filter, which spans \SIrange{590}{880}{\nano\meter}, in ``fastPol'' mode \citep{schmid_spherezimpol_2018}. The central PSF is saturated in these images to enable high dynamic range images without a coronagraph for diffraction control. The 2015 epoch was first presented in \citet{bertrang_hd_2018} and the 2018 epoch was first presented in \citet{bertrang_moving_2020}.

Lastly, we obtained the high-resolution ALMA \SI{1.3}{\milli\meter} continuum images first presented in \citet{perez_dust_2019}. We used the MEM deconvolved images, with a synthesized beam size of \SI{27}{\mas}$\times$\SI{20}{\mas} and an error of \SI{14.7}{\micro Jy/beam}.



\section*{Acknowledgements}
First and foremost, we wish to recognize and acknowledge the significant cultural role and reverence that the summit of Maunakea has always had within the Indigenous Hawaiian community. We are grateful and thank the community for the privilege of conducting observations from this sacred Mauna.

We thank telescope operators TODO for their support during the observations used in this work.

This research was funded by the Heising-Simons Foundation through grant \#2020-1823. Based on data collected at Subaru Telescope, which is operated by the National Astronomical Observatory of Japan. The development of SCExAO is supported by the Japan Society for the Promotion of Science (Grant-in-Aid for Research \#23340051, \#26220704, \#23103002, \#19H00703, \#19H00695, and \#21H04998), the Subaru Telescope, the National Astronomical Observatory of Japan, the Astrobiology Center of the National Institutes of Natural Sciences, Japan, the Mt Cuba Foundation and the Heising-Simons Foundation.

\bibliography{references}

\end{document}
