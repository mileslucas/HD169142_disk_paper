\section{Introduction\label{sec:introduction}}

\begin{figure}
    \centering
    \script{plot_disk_schematic.py}
    \includegraphics[width=\columnwidth]{figures/HD169142_schematic.pdf}
    \caption{Schematic diagram of the HD169142 transitional disk. The black contours are based on the ALMA 1.3mm continuum data which trace out an inner disk (B0) a close inner ring (B1) and three outer rings (B2-B4). The red. contours are based on visible to near-IR scattered-light images, separated into an inner and outer ring.\label{fig:disk_schematic}}
\end{figure}

\begin{deluxetable}{lcll}
    \centering
    \tablecaption{HD 169142 adopted parameters.\label{tbl:system}}
    \tablehead{
        \colhead{Parameter} &
        \colhead{Value} & 
        \colhead{Unit} & 
        \colhead{Ref.}
    }
    \startdata
    \cutinhead{System parameters}
    Distance & \num{114.87\pm0.35} & \si{pc} & [1] \\
    $M_\star$ & 1.65 & $M_\odot$ & \\
    \cutinhead{Disk parameters}
    Inclination & \num{12.5} & deg &  \\
    Position angle & \num{5} & deg &  \\
    \enddata
    \tablerefs{[1]: \citealt{gaia_collaboration_gaia_2021}}
\end{deluxetable}