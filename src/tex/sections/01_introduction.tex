\section{Introduction\label{sec:introduction}}


Exoplanet detection and characterization is one of the frontiers in modern observational astronomy, driven by cutting-edge instrumentation. In recent decades, thousands of exoplanets have been discovered, revealing a great diversity of planetary systems and architectures. However, the understanding of how these systems formed and evolved remains limited, with discrepancies between the observed planets and theories of planet formation. 

Direct observations of protoplanetary disks offer the best way to study planet formation by resolving features like spirals, ring gaps, and clumps \citep{andrews_observations_2020,benisty_optical_2023}. These features indicate over- and under-densities of disk material, which can be signatures of planet formation.

Major surveys have already detected most visibly bright disks ($m_{\text R}<$12) in nearby star-forming regions ($d <$ \SIrange{150}{400}{\parsec}; e.g., \citealt{avenhaus_disks_2018,garufi_disks_2020,rich_gemini-lights_2022,ren_protoplanetary_2023,valegard_sphere_2024,ginski_sphere_2024,garufi_sphere_2024}). Further advancements in disk science require shifting from detection to in-depth characterization.

High-contrast polarimetric imaging is the most powerful tool for studying planet-forming disks at optical to near-infrared wavelengths \citep{benisty_optical_2023}. This technique uses adaptive optics correction, large-diameter telescopes, coronagraphs for starlight suppression, polarization optics, and low-noise detectors to achieve the angular resolution and sensitivity required for detecting faint disk structures close to the intense diffracted light of the star.

The key advantage of polarimetry is that starlight reflected off disk material is partially polarized (20\% - 70\% at peak), whereas the star is unpolarized. Using a polarimeter to split and then subtract orthogonal polarization states cancels out the unpolarized stellar light to reveal the disk, known as polarimetric differential imaging (PDI; \citealt{kuhn_imaging_2001}). This cancellation attenuates the stellar signal without biases that alter the morphology and photometry of the disk, unlike typical point-spread function (PSF) subtraction techniques \citep{soummer_detection_2012,benisty_optical_2023}.

In this paper, we present dynamical analysis of polarimetric images of the planet-forming disk HD169142. We will present a combination of archival PDI images and new observations with scexao (\autoref{sec:observations}). We present the observations 