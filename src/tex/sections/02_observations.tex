\section{Observations\label{sec:observations}}

In this work, we present a combination of new observations of HD 169142 using SCExAO alongside previously published, archival datasets from various high-contrast instruments. All observations are presented in the observing log (\autoref{tbl:obslog}).


\begin{deluxetable*}{llllccccccl}
\tabletypesize{\footnotesize}
\tablehead{
    \colhead{Date} &
    \multirow{2}{*}{Telescope} &
    \multirow{2}{*}{Instrument} &
    % \multirow{2}{*}{Object} &
    \multirow{2}{*}{Filter} &
    \colhead{IWA} &
    \colhead{pix. scale} &
    \colhead{DIT} &
    \colhead{$T_\mathrm{exp}$} &
    \colhead{Seeing} &
    \multirow{2}{*}{Ref.} \vspace{-0.75em} \\
    \colhead{(UTC)} & % date
    & % inst
    & % object
    & % filter
    \colhead{(mas)} & % iwa
    \colhead{(mas/pix)} & % pxscale
    \colhead{(\si{\second})} & % dit
    \colhead{(\si{min})} & %texp
    \colhead{('')} & % seeing
     & % reference
}
\tablecaption{Chronologically-ordered observing log.\label{tbl:obslog}}
\startdata
\formatdate{26}{07}{2012} & VLT & NACO & H & sat. & 27 & 45 & 69 & $\sim$1.04 & [1] \\
\formatdate{25}{04}{2014} & Gemini-S & GPI & J & 184 & 14.14 & 29.1 & 62 & \num{0.85\pm0.27} & [2] \\
\formatdate{03}{05}{2015} & VLT & IRDIS & J & 80 & 12.263 & 16 & 53 & $\sim$0.9 & [3] \\ 
\formatdate{10}{07}{2015} & VLT & ZIMPOL & VBB & sat. & 3.6 & 10 & 56 & 0.75 & [4] \\
% \formatdate{18}{09}{2017} & ALMA & - &  \SI{1.3}{\milli\meter} & - & 5 & - & - & - & [6] \\
\formatdate{15}{07}{2018} & VLT & ZIMPOL & VBB & sat. & 3.6 & 4.4 & 53 & 0.51 & [5]  \\
\formatdate{06}{09}{2021} & VLT & IRDIS & Ks & 200 & 12.265 & 16 & 34 & $\sim$0.5 & [6] \\
\formatdate{3}{6}{2023} & Subaru & CHARIS & JHK & 113 & 15.16 & 60 & 52 & \num{0.82\pm0.15}$^a$ & \\ 
\formatdate{4}{6}{2023} & Subaru & CHARIS & JHK & 113 & 15.16 & 60 & 40 &  & \\ 
% \formatdate{7}{7}{2023} & Subaru & VAMPIRES &  MBI & 109 & 5.9 & 1 & 50 & \num{0.83\pm0.21} & [9] \\
\formatdate{29}{7}{2024} & Subaru & VAMPIRES & MBI & 54 & 5.9 & 2 & 108 & \num{0.6\pm0.14}$^a$ & \\
\enddata
\tablecomments{IWA: inner working angle of coronagraph, if present, or ``sat.'' for saturated non-coronagraphic observations. DIT: detector integration time for each frame. $T_\text{exp}$ total exposure time. (a): Seeing estimates, where present, were retrieved from the mean and standard deviation DIMM from the Maunakea Weather Center archive (\url{http://mkwc.ifa.hawaii.edu/current/seeing/index.cgi}).}
\tablerefs{[1]: \citealt{quanz_gaps_2013}, [2]: \citealt{monnier_polarized_2017}, [3]: \citealt{pohl_circumstellar_2017}, [4]: \citealt{bertrang_hd_2018}, [5]: \citealt{bertrang_moving_2020}, [8]: \citealt{ren_protoplanetary_2023}}


\end{deluxetable*}
\subsection{SCExAO/CHARIS Observations\label{sec:obs_charis}}

We observed HD 169142 over two nights in 2023 using the CHARIS integral-field spectrograph in broadband PDI mode \citep{groff_charis_2015,groff_first_2017}. The combination of a dispersing prism and microlens array creates spectra across the field of view from J to K band (\SIrange{1.12}{2.4}{\micro\meter}) and the Wollaston prism splits the ordinary and extraordinary beams which are simultaneously imaged on the detector. A rectangular \ang{;;2}x\ang{;;1} field stop is used to prevent the two orthogonal beams from overlapping creating a distinct wedge in derotated and collapsed data. An upstream half-wave plate (HWP) is modulated to measure the linear Stokes parameters ($Q$ and $U$) and correct for instrumental polarization. We used the \SI{113}{mas} IWA Lyot coronagraph with calibration ``satellite spot'' speckles for alignment and photometric calibration \citep{sahoo_precision_2020}.

Spectral cubes were extracted from each night's data using \texttt{CHARIS-DEP} \citep{brandt_data_2017}. Then, the spectral cubes were post-processed with sky subtraction, image registration, spectrophotometric calibration, and PDI using the \texttt{CHARIS-DPP} \citep{currie_subaruscexao_2017,lawson_high-contrast_2021}. The data was derotated and a Mueller-matrix correction was applied to the resulting Stokes Q and U images for each wavelength slice \citep{joost_t_hart_full_2021}. The individual Stokes Q and U images from both nights were combined, assuming negligible disk motion. Finally, for the Q and U images all wavelength slices, excluding the telluric absorption bands, were combined with an outlier-resilient mean. Some of the individual Q and U frames show that the coronagraphic mask was misaligned and partially obscuring the inner ring to the east. This only occurs in a few frames but does appear as a slight dip in intensity in the inner ring profiles. We calculated the azimuthal Stokes $Q_\phi$ and $U_\phi$ \citep{schmid_limb_2006,monnier_multiple_2019},
\begin{align}
\begin{split}
    \label{eqn:az_stokes}
    Q_\phi &= -Q\cos{\left(2\phi\right)} - U\sin{\left(2\phi\right)} \\
    U_\phi &= -Q\sin{\left(2\phi\right)} + U\cos{\left(2\phi\right)},
\end{split}
\end{align}
where
\begin{equation}
    \phi = \arctan{\left( \frac{x_\star - x}{y - y_\star} \right)} + \phi_0,
\end{equation}
the angle East of North plus any calibration offsets.

\subsection{SCExAO/VAMPIRES Observations\label{sec:obs_vampires}}

We observed HD 169142 in 2024 using SCExAO/VAMPIRES \citep{norris_vampires_2015,lucas_visible-light_2024-1}. Our observations utilized the new multiband imaging mode (MBI), which produces four broadband images simultaneously. We used the slow detector readout mode for low read noise and the ``slow'' polarimetry mode, which only uses the upstream HWP for modulation to correct for instrumental polarization. We used the \SI{54}{mas} IWA Lyot coronagraph with calibration speckles for alignment and photometric calibration.

We processed the data using the VAMPIRES data processing pipeline\footnote{\url{https://github.com/scexao-org/vampires_dpp}}. The processing steps included dark subtraction, sub-cropping the four MBI fields, frame registration, frame selection, coadding, photometric calibration, and polarimetric differential imaging (PDI). We registered the frames using sub-pixel phase cross-correlation \citep{guizar-sicairos_efficient_2008} with the mean frame of each data cube. We discarded the lowest 25\% of frames from each cube according to the estimated Strehl ratio before median-collapsing the cube into a single frame. The coadded frames were grouped according to their HWP angle, with incomplete groupings discarded. We performed double-difference imaging with a Mueller-matrix correction of each group to produce calibrated Stokes I, Q, and U frames for each wavelength. The four wavelengths were mean-combined before calculating the $Q_\phi$ and $U_\phi$ images.

\subsection{Archival Data\label{sec:obs_archive}}

The oldest dataset is the H-band polarimetric observation on the Very Large Telescope (VLT) Nasmyth Adaptive Optics System Near-Infrared Imager and Spectrograph (NACO; \citealp{quanz_very_2011}) presented in \citet{quanz_gaps_2013}. The data is non-coronagraphic, however, the central star is saturated and masked out in the images. We used the data presented in \citet{regt_polarimetric_2024} from the \texttt{PIPPIN} pipeline, which uses modern polarimetric data processing techniques in a consistent, purpose-built package for NACO.

The Gemini Planet Imager (GPI; \citealt{macintosh_first_2014}) observed HD 169142 in 2014 in J-band non-dispersed polarimetric mode \citep{perrin_polarimetry_2015}. A 184 mas IWA apodized Lyot coronagraph (APLC) was used for diffraction control. The data was originally presented in \citet{monnier_polarized_2017} and we use the data products from \citet{rich_gemini-lights_2022}.

The Spectro-Polarimetic High contrast imager for Exoplanets REsearch (SPHERE; \citealp{beuzit_sphere_2019}) has observed HD 169142 on numerous occasions with the InfraRed Dual Imager and Spectrograph (IRDIS; \citealp{dohlen_infra-red_2008}) and the Zurich IMaging POLarimeter (ZIMPOL; \citealp{schmid_spherezimpol_2018}). In 2015 it was observed with IRDIS in J-band polarimetric mode \citep{boer_polarimetric_2020} with an \SI{80}{\mas} IWA APLC, first presented in \citet{pohl_circumstellar_2017}. The data was processed with \texttt{IRDAP} \citep{holstein_polarimetric_2020} using the calibrated Mueller-matrix correction. HD 169142 was also observed in 2021 in Ks-band polarimetric mode with a \SI{200}{\mas} IWA APLC, first presented in \citet{ren_protoplanetary_2023}. The first ZIMPOL observation was in 2015 and 2018 using the very broadband (VBB) filter, which spans \SIrange{590}{880}{\nano\meter}, in ``fastPol'' mode \citep{schmid_spherezimpol_2018}. The central PSF is saturated in these images to enable high dynamic range images without a coronagraph for diffraction control. The 2015 epoch was first presented in \citet{bertrang_hd_2018} and the 2018 epoch was first presented in \citet{bertrang_moving_2020}.

Lastly, we obtained the high-resolution ALMA \SI{1.3}{\milli\meter} continuum images first presented in \citet{perez_dust_2019}. We used the MEM deconvolved images, with a synthesized beam size of \SI{27}{\mas}$\times$\SI{20}{\mas} and an error of \SI{14.7}{\micro Jy/beam}.

