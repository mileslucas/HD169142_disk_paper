\section{Observations\label{sec:observations}}

In this work, we present a combination of new observations of HD 169142 using SCExAO/VAMPIRES alongside previously published, archival datasets from various high-contrast instruments. All observations are presented in the observing log (\autoref{tbl:obslog}).


\begin{deluxetable*}{llllccccccl}
\tabletypesize{\footnotesize}
\tablehead{
    \colhead{Date} &
    \multirow{2}{*}{Telescope} &
    \multirow{2}{*}{Instrument} &
    % \multirow{2}{*}{Object} &
    \multirow{2}{*}{Filter} &
    \colhead{IWA} &
    \colhead{pix. scale} &
    \colhead{DIT} &
    \colhead{$T_\mathrm{exp}$} &
    \colhead{Seeing} &
    \multirow{2}{*}{Ref.} \vspace{-0.75em} \\
    \colhead{(UTC)} & % date
    & % inst
    & % object
    & % filter
    \colhead{(mas)} & % iwa
    \colhead{(mas/pix)} & % pxscale
    \colhead{(\si{\second})} & % dit
    \colhead{(\si{min})} & %texp
    \colhead{('')} & % seeing
     & % reference
}
\tablecaption{Chronologically-ordered observing log.\label{tbl:obslog}}
\startdata
\formatdate{26}{07}{2012} & VLT & NACO &  H & - & 27 & & & & [1] \\
\formatdate{25}{04}{2014} & Gemini-S & GPI &  J & 184 & 14.14 & 29.1 & 62 & \num{0.85\pm0.27} & [2,3] \\
\formatdate{03}{05}{2015} & VLT & IRDIS &  J & 80 & 12.25 & & & 0.9 & [4] \\ 
\formatdate{10}{07}{2015} & VLT & ZIMPOL &  VBB & - & 3.6 & & & & [5] \\
\formatdate{18}{09}{2017} & ALMA & - &  \SI{1.3}{\milli\meter} & - & 5 & - & - & - & [6] \\
\formatdate{15}{07}{2018} & VLT & ZIMPOL &  VBB & - & 3.6 & & & & [7]  \\
\formatdate{06}{09}{2021} & VLT & IRDIS &  Ks & 200 & 12.25 & 16 & 34 & & [8] \\
\formatdate{1}{6}{2023} & Subaru & CHARIS &  JHK & 113 & 15.16 &  &  &  & \\ 
\formatdate{2}{6}{2023} & Subaru & CHARIS &  JHK & 113 & 15.16 &  &  &  & \\ 
\formatdate{3}{6}{2023} & Subaru & CHARIS &  JHK & 113 & 15.16 &  &  &  & \\ 
\formatdate{4}{6}{2023} & Subaru & CHARIS &  JHK & 113 & 15.16 &  &  &  & \\ 
% \formatdate{7}{7}{2023} & Subaru & VAMPIRES &  MBI & 109 & 5.9 & 1 & 50 & \num{0.83\pm0.21} & [9] \\
\formatdate{29}{7}{2024} & Subaru & VAMPIRES & MBI & 54 & 5.9 & 2 & 108 & \num{0.6\pm0.14} & \\
\enddata
\tablecomments{IWA: inner working angle of coronagraph, if present. DIT: detector integration time for each frame. $T_\text{exp}$ total exposure time. Seeing estimates, where present, were retrieved from the reference publication for archival data and the mean and standard deviation DIMM from the Maunakea Weather Center archive for the VAMPIRES observations.}
\tablerefs{[1]: \citealt{quanz_gaps_2013}, [2]: \citealt{monnier_polarized_2017}, [3]: \citealt{rich_gemini-lights_2022}, [4]: \citealt{pohl_circumstellar_2017}, [5]: \citealt{bertrang_hd_2018}, [6]: \citealt{perez_dust_2019}, [7]: \citealt{bertrang_moving_2020}, [8]: \citealt{ren_protoplanetary_2023}}
\end{deluxetable*}

\subsection{SCExAO/CHARIS Observations\label{sec:obs_charis}}

We observed HD 169142 over four nights in 2023 using the CHARIS integral-field spectrograph \citep{groff_charis_2015,groff_first_2017} in broadband PDI mode. In this mode, the combination of a dispersing prism and microlens array creates spectra across the field of view from J to K band. With the Wollaston prism both the ordinary and extraordinary beams are imaged and a rectangular \ang{;;2}x\ang{;;1} field stop prevents overlap of the beams. An upstream half-wave plate (HWP) is modulated to measure the linear Stokes parameters and correct for instrumental polarization. We used the \SI{113}{mas} IWA Lyot coronagraph with calibration ``satellite spot'' speckles for alignment and photometric calibration \citep{sahoo_precision_2020}.

Spectral cubes were extracted from each night's data using the \texttt{CHARIS-DRP}. Then, the spectral cubes were post-processed to do sky subtraction, image registration, spectrophotometric calibration, and PDI using the \texttt{CHARIS-DPP}. The data was derotated and a Mueller-matrix correction was applied to the resulting Stokes Q and U images for each wavelength slice \citep{joost_t_hart_full_2021}. The individual Stokes Q and U images from all four nights were combined, assuming there will be negligible disk motion over four days. Finally, for the Q and U images all wavelength slices, excluding the telluric absorption bands, were mean-combined. We calculated the azimuthal Stokes $Q_\phi$ and $U_\phi$ \citep{schmid_limb_2006,monnier_multiple_2019},
\begin{align}
\begin{split}
    \label{eqn:az_stokes}
    Q_\phi &= -Q\cos{\left(2\phi\right)} - U\sin{\left(2\phi\right)} \\
    U_\phi &= Q\sin{\left(2\phi\right)} - U\cos{\left(2\phi\right)},
\end{split}
\end{align}
where
\begin{equation}
    \phi = \arctan{\left( \frac{x_\star - x}{y - y_\star} \right)} + \phi_0.
\end{equation}

\subsection{SCExAO/VAMPIRES Observations\label{sec:obs_vampires}}

We observed HD 169142 in 2024 using VAMPIRES \citep{norris_vampires_2015,lucas_visible-light_2024-1} on the Subaru Coronagraphic Extreme Adaptive Optics instrument \citep{jovanovic_subaru_2015}. Our observations utilized the new multiband imaging mode (MBI), which produces four broadband images simultaneously. We used the slow detector readout mode for low read noise and the ``slow'' polarimetry mode, which only uses the upstream HWP for modulation to correct for instrumental polarization. We used the \SI{54}{mas} IWA Lyot coronagraph with calibration speckles for alignment and photometric calibration.

We processed the data using the VAMPIRES data processing pipeline\footnote{\url{https://github.com/scexao-org/vampires_dpp}}. The processing steps included dark subtraction, sub-cropping the four MBI fields, frame registration, frame selection, coadding, photometric calibration, and polarimetric differential imaging (PDI). We registered the frames using sub-pixel phase cross-correlation \citep{guizar-sicairos_efficient_2008} with the mean frame of each data cube. We discarded the lowest 25\% of frames from each cube according to the estimated Strehl ratio before median-collapsing the cube into a single frame. The coadded frames are grouped according to their HWP angle, with incomplete groupings discarded. We performed double-difference imaging with a Mueller matrix correction of each group to produce calibrated Stokes I, Q, and U frames for each wavelength. The four wavelengths were mean-combined before calculating the $Q_\phi$ and $U_\phi$ images.

\subsection{Archival Data\label{sec:obs_archive}}

HD 169142 was observed by VLT/NACO in 2012 using its H-band polarimetric mode \citep{quanz_very_2011}. The original data is described in detail in \citet{quanz_gaps_2013}; however, we have used the newly-processed data products from the \texttt{PIPPIN} pipeline \citep{regt_polarimetric_2024}, which uses modern polarimetric data processing techniques in a consistent pipeline purpose-built for NACO. 

The Gemini Planet Imager (GPI; \citealt{macintosh_first_2014}) observed HD 169142 in 2014 using its J-band non-dispersed polarimetric mode. A 999 mas IWA apodized Lyot coronagraph was used for diffraction control. The data is presented in \citet{monnier_polarized_2017} and \citet{rich_gemini-lights_2022}. 

VLT/SPHERE \citep{beuzit_sphere_2019} has observed HD 169142 on numerous occasions with both IRDIS \citep{boer_polarimetric_2020,holstein_polarimetric_2020} and ZIMPOL \citep{schmid_spherezimpol_2018}. In 2015 it was observed with IRDIS at J-band \citep{pohl_circumstellar_2017} and ZIMPOL in their VBB filter, which roughly spans R- to I-band \citep{bertrang_hd_2018,tschudi_quantitative_2021}. In 2018, it was observed by ZIMPOL, again, in the VBB filter \citep{bertrang_moving_2020}. In 2021, HD 169142 was observed with IRDIS at K-band using star-hopping \citep{ren_protoplanetary_2023}.

Lastly, we obtained the high-resolution ALMA \SI{1.3}{\milli\meter} continuum images presented in \citet{perez_dust_2019}.

